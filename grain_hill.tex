%% Copernicus Publications Manuscript Preparation Template for LaTeX Submissions
%% ---------------------------------
%% This template should be used for copernicus.cls
%% The class file and some style files are bundled in the Copernicus Latex Package which can be downloaded from the different journal webpages.
%% For further assistance please contact the Copernicus Publications at: publications@copernicus.org
%% http://publications.copernicus.org


%% Please use the following documentclass and Journal Abbreviations for Discussion Papers and Final Revised Papers.


%% 2-Column Papers and Discussion Papers
\documentclass[esurf, manuscript]{copernicus}





%% \usepackage commands included in the copernicus.cls:
%\usepackage[german, english]{babel}
%\usepackage{tabularx}
%\usepackage{cancel}
%\usepackage{multirow}
%\usepackage{supertabular}
%\usepackage{algorithmic}
%\usepackage{algorithm}
%\usepackage{amsthm}
%\usepackage{float}
%\usepackage{subfig}
%\usepackage{rotating}


\begin{document}

\title{A lattice grain model of hillslope evolution}


% \Author[affil]{given_name}{surname}

\Author[1]{Gregory E.}{Tucker}
\Author[]{}{}
\Author[]{}{}

\affil[1]{Cooperative Institute for Research in Environmental Science (CIRES) and Department of Geological Sciences, University of Colorado, Boulder, CO 80305 USA}
\affil[]{ADDRESS}

%% The [] brackets identify the author with the corresponding affiliation. 1, 2, 3, etc. should be inserted.



\runningtitle{TEXT}

\runningauthor{TEXT}

\correspondence{NAME (EMAIL)}



\received{}
\pubdiscuss{} %% only important for two-stage journals
\revised{}
\accepted{}
\published{}

%% These dates will be inserted by Copernicus Publications during the typesetting process.


\firstpage{1}

\maketitle



\begin{abstract}
TEXT
\end{abstract}



\introduction  %% \introduction[modified heading if necessary]

Hillslopes take on a rich variety of forms. Their profile shapes may be convex-upward, concave-upward, planar, or some combination of these. Some slopes are completely mantled with soil, whereas others are bare rock, and still others draped in a discontinuous layer of mobile regolith. The processes understood to be responsible for shaping them are equally varied, ranging from disturbance-driven creep to dissolution to large-scale mass movement events.

Considerable research has been devoted to understanding the evolution of soil-mantled slopes that are primarily governed by disturbance-driven creep, such as down-slope soil transport by biotic and abiotic soil-mixing processes. As a result, the geomorphology community has mathematical models that account well for observed slope forms and patterns of regolith thickness \citep[e.g.,][]{roering2008how}. Furthermore, stochastic-transport theory provides a mechanistic link between the statistics of particle motion, the resultant average rates of downslope transport, and the emergence of convex-upward, soil-mantled slope forms \citep{culling,furbish}.

One gap that remains, however, lies in understanding steep, rocky slopes. ``Rocky'' implies slopes that lack a continuous soil cover; here, transport laws that assume the existence such a cover no longer apply. ``Steep'' implies angles approaching or exceeding the effective angle of repose for loose, granular material, so that ravel may be an important transport model \citep[e.g.][]{gabet,lamb,stock} and particles have the potential to fall as soon as they are released from bedrock. This type of relative fast, long-distance transport does not fit comfortably in the framework of standard diffusion-based models of hillslope soil transport, which derive from an underlying assumption that the characteristic length scale of motion is short relative to the length of the slope [REFS]. 

Rocky slopes are rarely completely barren; more commonly, they have a patchy cover of loose scree, which could either retard rock weathering by shielding the rock surface from moisture or temperature fluctuations, or enhance it by trapping water and allowing limited plant growth. A discontinuous cover does not fit easily within the popular exponential-decay regolith-production models [SOME REF], which assume an essentially continuous soil mantle.

An additional issue, which pertains to both rocky and soil-mantled slopes, is the connection between sediment motion at the scale of motion ``events'', and longer-term average sediment flux, which forms the basis for continuum models of hillslope evolution. Recent theoretical and experiment work has begun to forge a mechanistic connection between these scales [REFS Furbish especially, building on earlier work by Culling]. However, the community's resources for computational analysis of particle-level dynamics remain limited [REFS], lagging behind developments in the understanding of fluvial sediment transport [REFS Schmeeckle etc.].

To further our understanding of how grain-level weathering and transport translate into hillslope evolution, both for hillslopes in general and rocky slopes in particular, it would be useful to have a computational framework with which to conduct experiments. Ideally, such a framework should be sophisticated enough to capture the essence of weathering and granular mechanics, while remaining simple enough to involve only a small number of parameters and providing reasonable computational efficiency.

Our aim in this paper is to describe one such computational framework, test whether it is capable of reproducing commonly observed hillslope-profile forms, and examine how its parameters relate to the bulk-behavior parameters used in conventional continuum models of soil creep and regolith production. The model uses a pairwise, continuous-time stochastic (CTS) approach to combine a lattice-grain model [REFS] with rules for stochastic bedrock-to-regolith conversion (``weathering'') and disturbance of surface regolith particles.

We begin with ... [outline sections of the paper]






\section{Background}

Do we go all the way back to Gilbert? probably, but quite briefly. 

gilbert reasoned that the rate of release of rock or saprolite to disaggregated material should depend on the thickness of the overlying regolith cover, because XYZ. this was later codified into the popular inverse-exponential and ``humped curve'' formulas [ahnert, heimsath, anderson, etc]. consistent with cosmo nuclides [heim, small, etc]. lacks mechanistic basis.

Davis and Gilbert enunciated the view of convex soil-mantled slopes, turned into math by culling, who also nodded to underlying probabilistic basis.

diffusion theory captures convex slopes, and is consistent with cosmos [mckean, small, ...]

modified to account for accelerated motion, nonlinear, andrews and bucknam, howard, roering

also modified for account for depth dependence

process blind; some specific formulations for particular types of disturbance process (e.g., frost creep) (just mentioning in passing)

Furbish work that relates bulk flux of sediment to the statistical mechanics of grain motion, leading to diffusion-like principles. Foufoula nonlocal and fractional calculus.

hillslope cellular and particle models review: jyotsna and haff, tucker and bradley, roering and ?.

meanwhile, in the granular mechanics community, a whole variety of cellular models (see refs in T et al 2016)

one of our aims in this paper is to forge a link between the lattice-grain models used in the granular mechanics community, and theories for hillslope evolution in the geomorphology community.


\section{Model Description}

Lattice-grain model 

works on pairwise CTS principle; explain briefly

brief overview of LG rules and framework; refer to Tucker et al. 2016

Some kind of demonstrations / examples. Silo again? Sand-pile?

Weathering rule: rock turns to regolith

Disturbance rule: 

Uplift

Scaling and nondimensionalization


\section{Results}
TEXT


\section{Discussion}


\conclusions  %% \conclusions[modified heading if necessary]
TEXT




\appendix
\section{}    %% Appendix A

\subsection{}                               %% Appendix A1, A2, etc.


\authorcontribution{TEXT}

\begin{acknowledgements}
TEXT
\end{acknowledgements}


%% REFERENCES

%% The reference list is compiled as follows:

\begin{thebibliography}{}

\bibitem[AUTHOR(YEAR)]{LABEL}
REFERENCE 1

\bibitem[AUTHOR(YEAR)]{LABEL}
REFERENCE 2

\end{thebibliography}

%% Since the Copernicus LaTeX package includes the BibTeX style file copernicus.bst,
%% authors experienced with BibTeX only have to include the following two lines:
%%
%% \bibliographystyle{copernicus}
%% \bibliography{example.bib}
%%
%% URLs and DOIs can be entered in your BibTeX file as:
%%
%% URL = {http://www.xyz.org/~jones/idx_g.htm}
%% DOI = {10.5194/xyz}


%% LITERATURE CITATIONS
%%
%% command                        & example result
%% \citet{jones90}|               & Jones et al. (1990)
%% \citep{jones90}|               & (Jones et al., 1990)
%% \citep{jones90,jones93}|       & (Jones et al., 1990, 1993)
%% \citep[p.~32]{jones90}|        & (Jones et al., 1990, p.~32)
%% \citep[e.g.,][]{jones90}|      & (e.g., Jones et al., 1990)
%% \citep[e.g.,][p.~32]{jones90}| & (e.g., Jones et al., 1990, p.~32)
%% \citeauthor{jones90}|          & Jones et al.
%% \citeyear{jones90}|            & 1990



%% FIGURES

%% ONE-COLUMN FIGURES

%%f
%\begin{figure}[t]
%\includegraphics[width=8.3cm]{FILE NAME}
%\caption{TEXT}
%\end{figure}
%
%%% TWO-COLUMN FIGURES
%
%%f
%\begin{figure*}[t]
%\includegraphics[width=12cm]{FILE NAME}
%\caption{TEXT}
%\end{figure*}
%
%
%%% TABLES
%%%
%%% The different columns must be seperated with a & command and should
%%% end with \\ to identify the column brake.
%
%%% ONE-COLUMN TABLE
%
%%t
%\begin{table}[t]
%\caption{TEXT}
%\begin{tabular}{column = lcr}
%\tophline
%
%\middlehline
%
%\bottomhline
%\end{tabular}
%\belowtable{} % Table Footnotes
%\end{table}
%
%%% TWO-COLUMN TABLE
%
%%t
%\begin{table*}[t]
%\caption{TEXT}
%\begin{tabular}{column = lcr}
%\tophline
%
%\middlehline
%
%\bottomhline
%\end{tabular}
%\belowtable{} % Table Footnotes
%\end{table*}
%
%
%%% NUMBERING OF FIGURES AND TABLES
%%%
%%% If figures and tables must be numbered 1a, 1b, etc. the following command
%%% should be inserted before the begin{} command.
%
%\addtocounter{figure}{-1}\renewcommand{\thefigure}{\arabic{figure}a}
%
%
%%% MATHEMATICAL EXPRESSIONS
%
%%% All papers typeset by Copernicus Publications follow the math typesetting regulations
%%% given by the IUPAC Green Book (IUPAC: Quantities, Units and Symbols in Physical Chemistry,
%%% 2nd Edn., Blackwell Science, available at: http://old.iupac.org/publications/books/gbook/green_book_2ed.pdf, 1993).
%%%
%%% Physical quantities/variables are typeset in italic font (t for time, T for Temperature)
%%% Indices which are not defined are typeset in italic font (x, y, z, a, b, c)
%%% Items/objects which are defined are typeset in roman font (Car A, Car B)
%%% Descriptions/specifications which are defined by itself are typeset in roman font (abs, rel, ref, tot, net, ice)
%%% Abbreviations from 2 letters are typeset in roman font (RH, LAI)
%%% Vectors are identified in bold italic font using \vec{x}
%%% Matrices are identified in bold roman font
%%% Multiplication signs are typeset using the LaTeX commands \times (for vector products, grids, and exponential notations) or \cdot
%%% The character * should not be applied as mutliplication sign
%
%
%%% EQUATIONS
%
%%% Single-row equation
%
%\begin{equation}
%
%\end{equation}
%
%%% Multiline equation
%
%\begin{align}
%& 3 + 5 = 8\\
%& 3 + 5 = 8\\
%& 3 + 5 = 8
%\end{align}
%
%
%%% MATRICES
%
%\begin{matrix}
%x & y & z\\
%x & y & z\\
%x & y & z\\
%\end{matrix}
%
%
%%% ALGORITHM
%
%\begin{algorithm}
%\caption{�}
%\label{a1}
%\begin{algorithmic}
%�
%\end{algorithmic}
%\end{algorithm}
%
%
%%% CHEMICAL FORMULAS AND REACTIONS
%
%%% For formulas embedded in the text, please use \chem{}
%
%%% The reaction environment creates labels including the letter R, i.e. (R1), (R2), etc.
%
%\begin{reaction}
%%% \rightarrow should be used for normal (one-way) chemical reactions
%%% \rightleftharpoons should be used for equilibria
%%% \leftrightarrow should be used for resonance structures
%\end{reaction}
%
%
%%% PHYSICAL UNITS
%%%
%%% Please use \unit{} and apply the exponential notation


\end{document}
