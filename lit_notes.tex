\documentclass[12pt, oneside]{article}   	% use "amsart" instead of "article" for AMSLaTeX format
\usepackage{geometry}                		% See geometry.pdf to learn the layout options. There are lots.
\geometry{letterpaper}                   		% ... or a4paper or a5paper or ... 
%\geometry{landscape}                		% Activate for rotated page geometry
%\usepackage[parfill]{parskip}    		% Activate to begin paragraphs with an empty line rather than an indent
\usepackage{graphicx}				% Use pdf, png, jpg, or eps§ with pdflatex; use eps in DVI mode
								% TeX will automatically convert eps --> pdf in pdflatex		
\usepackage{amssymb}

%SetFonts

%SetFonts


\title{Notes on lit related to grain hill}
\author{GT}
%\date{}							% Activate to display a given date or no date

\begin{document}
\maketitle

\section*{Furbish et al.~(2009) JGR}

Split flux into product of depth, $h$, and ``flux density'' (a depth-averaged velocity, though they argue not really the same as speed) $\bar{\mathbf{q}}$: $\mathbf{q}_s = h \bar{\mathbf{q}}$

``we envision the en masse motion of a soil as arising from the collective quasi-random motions of soil particles and particle clumps, where the overall motion involves a net downward component that is gravitationally driven.''

Scattering and settling

[reminder to look at role of elastic vs inelastic proportions]

Settling depends on porosity (space into which to settle)

``Details of scattering motions, which generally are more complex, have a less critical role, so the analysis is inherently forgiving of any sins of omission or commission in our treatment of these motions.''

``the flux includes both advective and diffusive parts''

Separates motions into scattering and settling.

Formulates advection-dispersion FP. Novelty is activity probability and modes of motion.

$D$ sits INSIDE the derivatives! Basically, for each mode and direction, one gets:
\begin{equation}
q_x = c a M_1 u_1 - \frac{1}{2} \frac{\partial}{\partial x} ( c a M_1 D_1 ) +  c a M_2 u_2 - \frac{1}{2} \frac{\partial}{\partial x} ( c a M_2 D_2 )
\end{equation}
Here $c$ is concentration, $a$ is the fraction that are active, $M_i$ is the fraction traveling by model $i$ (scattering or settling), $u$ is the drift velocity, and $D$ is diffusivity.











%\subsection{}



\end{document}  